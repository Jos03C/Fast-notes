\documentclass[10pt,a4paper]{book}
\usepackage[english]{babel}
\usepackage[utf8x]{inputenc}
\usepackage[T1]{fontenc}
\usepackage{scribe}
\usepackage{listings}
\usepackage{eufrak}

\newenvironment{hints}{\textbf{Hints.}}{}

\Scribe{Zhuohua Shen}
\Lecturer{}
\LectureNumber{}
\LectureDate{Oct 2024}
\LectureTitle{Statistical Inference}

\lstset{style=mystyle}

\begin{document}
	\MakeScribeTop
	\tableofcontents

%#############################################################
%#############################################################
%#############################################################
%#############################################################

\chapter{Preliminary}\label{chap:premi}


\chapter{Statistical inference fundamentals}\label{chap:stat-inf}

References: most of the contents are from the undergraduate course STA3020 (by Prof. Jianfeng Mao in 2022-2023 T1, and Prof. Jiasheng Shi in 2023-2024 T2) and postgraduate course STAT5010 (by Kin Wai Keith Chan in 2024-2025 T1), with main textbook Casella and Berger \cite{casella2002statistical} 


\section{Statistical Models}
See Chapter 3 of \cite{casella2002statistical}. Suppose $X_i\simiid \bbP_*$, where $\bbP_*$ refers to the unknown \red{data generating process} (DGPg), we find $\widehat{\bbP}\approx \bbP_*$. A \red{statistical model} is a set of distributions $\scrF=\{\bbP_{\theta}:\theta\in\Theta\}$, where $\Theta$ is the \red{parameter space}. A \red{parametric model} is the model with $\dim(\Theta)<\infty$, while a \red{nonparametric model} satisfies $\dim(\Theta)=\infty$. 

\begin{defbox}
	\begin{definition}[\textbf{Exponential family}]\label{def:exp-family}
		A k-dimensional \red{exponential family} (EF) $\scrF=\{f_\theta:\theta\in\Theta\}$ is a model consisting of pdfs of the form
		\begin{equation}\label{eq:exp-family-pdf}
			f_\theta(x)=c(\theta)h(x)\exp\left\{\sum_{j=1}^k\eta_j(\theta)T_j(x)\right\}
		\end{equation}
	\end{definition}
	where $c(\theta),h(x)\geq 0$, $\Theta=\{\theta:c(\theta)\geq 0, \eta_j(\theta) \text{ being well defined for } 1\leq j\leq k\}$. Let $\eta_j=\eta_j(\theta)$, the \red{canonical form} is 
	\begin{equation}\label{eq:exp-family-can}
		f_\eta(x)=b(\eta)h(x)\exp\left\{\sum_{j=1}^k\eta_jT_j(x)\right\},
	\end{equation}
	\begin{itemize}[itemsep=2pt,topsep=0pt,parsep=0pt]
		\item $k$-dim \red{natural exponential family} (NEF): $\scrF'=\{f_{\eta}:\eta\in \Xi\}$; 
		\item \red{natural parameter} $\eta=(\eta_1,\ldots,\eta_k)^\TT$;
		\item \red{natural parameter space}: $\Xi=\{\eta\in\mathbb{R}^k:0<b(\eta)<\infty\}$; 
		\item the NEF $\scrF'$ is of \red{full rank} if $\Xi$ contains an open set in $\bbR^k$;
		\item the EF is a \red{curved exponential family} if $p=\dim(\Theta)<k$.
	\end{itemize}
\end{defbox}
\textbf{Properties of EF}:
\begin{itemize}[itemsep=2pt,topsep=0pt,parsep=0pt]
	\item Let $X\sim f_{\eta}$, where $\eta\in\Xi$ such that (i) $f_{\eta}$ is of the form \eqref{eq:exp-family-can} with $B(\eta)=-\log b(\eta)$, and (ii) $\Xi$ contains an open set in $\bbR^k$. Then, for $j,j'=1,\ldots,k$, $\bbE\{T_j(X)\}={\partial B(\eta)}/{\partial\eta_j}$ and $\Cov\{T_j(X),T_{j'}(X)\}={\partial^2B(\eta)}/\sbk{\partial\eta_j\partial\eta_{j'}}$.  
	\item \red{Stein's identity}:
\end{itemize}

\begin{defbox}
	\begin{definition}[\textbf{Location-scale family}]\label{def:ls-family}
		Let $f$ be a density. 
		\begin{itemize}
			\item A \red{location-scale family} is given by $\scrF=\{f_{\mu,\sigma}:\mu \in\bbR, \sigma\in\bbR^{++}\}$, where $f_{\mu,\sigma}(x)= f\left({(x-\mu)}/\sigma\right)/\sigma$.
			\item \red{location parameter}: $\mu$; \red{scale parameter}: $\sigma$; \red{standard density}: $f$;
			\item A \red{location family} is $\scrF=\{f_{\mu,1}:\mu\in\bbR\}$.
			\item A \red{scale family} is $\scrF=\{f_{0,\sigma}:\sigma\in\bbR^{++}\}$ 
		\end{itemize}
	\end{definition}
\end{defbox}
\textbf{Representation}: $X=\mu + \sigma Z$, $Z\sim f_{0,1}(\cdot)$. 
\begin{itemize}
	\item See some examples in Example 3.9, Keith's note 3, and Table 1 in Shi's note L1. 
	\item Transform between location parameter and scale parameter by taking log.
\end{itemize}


\begin{defbox}
	\begin{definition}[\textbf{Identifiable family}]\label{def:id-family}
		If $\forall \theta_1,\theta_2\in\Theta$ that    
		\begin{equation*}
			\theta_1\neq\theta_2\quad\Rightarrow\quad f_{\theta_1}(\cdot)\neq f_{\theta_2}(\cdot),
		\end{equation*}
		then $\scrF$ is said to be an \red{identifiable family}, or equivalently $\theta\in\Theta$ is \red{identifiable}. 
	\end{definition}
\end{defbox}

A typical feature of non-identifiable EF is that $p=\dim(\Theta)>k$. Typically,
\begin{itemize}
	\item $p<k$, curved (must).
	\item $p=k$, of full rank.
	\item $p>k$, non-identifiable.  
\end{itemize} 

\section{Principles of Data Reduction}\label{sec:prin-data-reduce}
\red{Statistics:} $T=T(X_{1:n})$, a function of $X_{1:n}$ and free of any unknown parameter.  

\subsection{Sufficiency Principle}\label{sec:prin-data-reduce-suff}
\textbf{Sufficiency principle}: If $T=T(X_{1:n})$ is a ``sufficient statistics'' for $\theta$, then any inference on $\theta$ will depend on $X_{1:n}$ only through $T$.

\begin{defbox}
	\begin{definition}[\textbf{Sufficient, minimal sufficient, ancillary, and complete statistics}]\label{def:stat-SS-MSS-ANS-CS}
		Suppose $X_{1:n}\simiid \bbP_{\theta}$, where $\theta\in\Theta$. Let $T=T(X_{1:n})$ be a statistic. Then $T$ is \red{sufficient} (SS) for $\theta$
		\begin{itemize}
			\item[$\Leftrightarrow$] (def) $[X_{1:n}\mid T=t]$ is free of $\theta$ for each $t$.
			\item[$\Leftrightarrow$] (technical lemma) $T(x_{1:n})=T(x_{1:n}')$ implies that $f_{\theta}(x_{1:n})/f_{\theta}(x_{1:n}')$ is free of $\theta$.
			\item[$\Leftrightarrow$] (Neyman-Fisher factorization theorem) $\forall\theta\in\Theta$, $x_{1:n}\in\scrX^n$, $f_\theta(x_{1:n})=A(t,\theta)B(x_{1:n})$.
		\item[$\Leftrightarrow$] Define $\Lambda(\theta',\theta''\mid x_{1:n}):=f_{\theta'}(x_{1:n})/f_{\theta''}(x_{1:n})$. $\forall \theta',\theta''\in\Theta$, $\exists$ function $C_{\theta',\theta''}$ such that $\Lambda(\theta',\theta''\mid x_{1:n})=C_{\theta',\theta''}(t)$, for all $x_{1:n}\in\scrX^n$ where $t=T(x_{1:n})$.    
		\end{itemize}
	$T$ is \red{minimal sufficient} (MSS) for $\theta$
	\begin{itemize}
		\item[$\Leftrightarrow$] (def) (1) $T$ is a SS for $\theta$; (2) $T=g(S)$ for any other SS $S$.
		\item[$\Leftrightarrow$] (1) $T$ is a SS for $\theta$; (2) $S(x_{1:n})=S(x_{1:n}')$ implies $T(x_{1:n})=T(x_{1:n}')$ for any SS $S$.   
		\item[$\Leftrightarrow$] (Lehmann-Scheffé theorem) $\forall x_{1:n},x_{1:n}'\in\scrX^n$, $f_{\theta}(x_{1:n})/f_{\theta}(x_{1:n}')$ is free of $\theta$ $\Leftrightarrow$ $T(x_{1:n})=T(x_{1:n}')$.    
	\end{itemize}
	$A=A(X_{1:n})$ is \red{ancillary} (ANS) if the distribution of $A$ does not depend on $\theta$. 

	\noindent $T$ is \red{complete} (CS) if $\forall\theta\in\Theta$, $\bbE_{\theta}g(T)=0$ implies $\forall\theta\in\Theta$, $\bbP_{\theta}\{g(T)=0\}=1$. 
	\end{definition}
\end{defbox}
\noindent\textbf{Properties}
\begin{itemize}
	\item (Transformation) If $T=r(T')$, then (i) $T$ is SS $\Rightarrow$ $T'$ is SS; (ii) $T'$ is CS $\Rightarrow$ $T$ is CS; (iii) $r$ is one-to-one, then if one is SS/MSS/CS, then the another is.    
	\item (\red{Basu's Lemma}) $X_i\simiid\bbP_\theta$, $A$ is ANS and $T$ s CSS, then $A \indep T$.
	\item (\red{Bahadur's theorem}) $X_i\simiid\bbP_\theta$, if an MSS exists, then any CSS is also an MSS.
	\begin{itemize}
		\item Then if a CSS exists, then any MSS is also a CSS $\Rightarrow$ CSS=MSS.
		\item \red{All or nothing}: start with MSS $T$, check whether $T$ is CS. (i) Yes, it is both CSS and MSS, then the set of MSS=CSS; (ii) No, there is no CSS at all.  
	\end{itemize}
	\item (Exp-family) If $X_i\simiid f_{\eta}$ in \eqref{eq:exp-family-can}, then $T=(\sum_{i=1}^nT_1(X_i),\ldots,\sum_{i=1}^nT_k(X_i))$ is a SS, called \red{natural sufficient statistic}. If $\Xi$ contains an open set in $\bbR^k$ (i.e., $\scrF'$ is of full rank), then $T$ is MSS and CSS. 
\end{itemize}

\noindent\textbf{Proof techniques}
\begin{itemize}
	\item Prove $T$ is not sufficient for $\theta$: show if $\exists x_{1_n}, x_{1:n}'\in\calX^n$ and $\theta',\theta''\in\Theta$, such that $T(x_{1:n})=T(x_{1:n}')$ and $\Lambda(\theta',\theta''\mid x_{1:n})\neq \Lambda(\theta',\theta''\mid x_{1:n}')$.  
	\item Prove $A$ is an ANS: consider location-scale representation.
	\item Prove $T$ is a CS: use definition or take $\rmd\bbE_{\theta}g(T)/\rmd\theta=0$. 
	\item Disprove $T$ is CS: 
	\begin{itemize}
		\item Construct an ANS $S(T)$ based on $T$, then $\bbE S(T)$ is free of $\theta$, then $g(T)=S(T)-\bbE S(T)$ is free of $\theta$ but $g(T)\neq 0$ w.p.1. 
		\item (Cancel the 1st moment) Find two unbiased estiamtors for $\theta$ as a function of $T$. E.g., $X_1,X_2\simiid \mathrm{N}(\theta,\theta^2)$, $T=(X_1,X_2)$, $g(T)=X_1-X_2\sim\mathrm{N}(0,2\theta^2)$. 
	\end{itemize}
	
\end{itemize}


\begin{remark}\label{rmk:SS-MSS-ANS-CS}
	\begin{itemize}
		\item ANS $A$ is useless on its own, but useful together with other information. 
		\item $\bbP(A(\X)\mid \theta)$ is free of $\theta$, but for non-SS $T$, $\bbP(A(\X)\mid T(\X))$ is not necessarily free of $\theta$. 
	\end{itemize}
\end{remark}

\subsection{Likelihood principle}\label{sec:prin-data-reduce-lik}




\chapter{Multivariate Inference Fundamentals}\label{chap:multi}
Reference: 
\begin{itemize}
	\item Robb J. Muirhead - Aspects of multivariate statistical theory \cite{muirhead1982aspects}.
	\item CUHK STAT4002 - Applied Multivariate Analysis (2023 Spring), by Zhixiang Lin.
\end{itemize}

\section{Random vectors and distributions}\label{sec:random_vector}
\begin{defbox}
	\begin{definition}\label{def:random_vector_moments}
		Let $\x=(x_1,\ldots, x_p)^\TT\in\bbR^p$ be a random vector, 
		\begin{itemize}
			\item Mean $\bbE \x = \bmu=(\bbE x_1,\ldots,\bbE x_p)^\TT=(\mu_j)$.
			\item Covariance matrix $\Var(\x)=\Cov(\x)=\Sigma=\bbE[(\x-\bbE\x)(\x-\bbE\x)^\TT]=\bbE \x\x^\TT - \bbE \x \bbE \x^\TT = (\sigma_{ij})$, $\Sigma\succeq \0$.
			\item Correlation matrix $R=D^{-1/2}\Sigma D^{-1/2}$, where $D=\diag(\sigma_{11},\ldots,\sigma_{pp})$. We have $R_{ij}=\rho_{ij}=\sigma_{ij}/(\sqrt{\sigma_{ii}}\sqrt{\sigma_{jj}})$.
			\item If $\y\in\bbR^{q}$ random vector, then $\Cov(\x,\y)=\bbE[(\x-\bbE\x)(\y-\bbE\y)^\TT]=\bbE\x\y^\TT-\bbE\x\bbE\y^\TT\in\bbR^{p\times q}$.  
		\end{itemize} 
		If $\Z=(z_{ij})\in\bbR^{p\times q}$ is a random matrix,
		\begin{itemize}
			\item $\bbE\Z=(\bbE z_{ij})$. 
		\end{itemize} 
	\end{definition}
\end{defbox}

\begin{thmbox}
	\begin{proposition}\label{prop:random_vector_moments}
		Let $\x\in\bbR^p$ be a random vector, $\a,\b\in\bbR^p$ be vectors, $A\in\bbR^{r_1\times p},B\in\bbR^{r_2\times p}$ be matrices,  
		\begin{itemize}
			\item $\bbE \a^\TT\x=\a^\TT \bbE\x$, $\Var(\a^\TT \x)=\a^\TT\Sigma\a$, and $\Cov(\a^\TT\x,\b^\TT\x)=\a^\TT \Sigma\b$.
			\item $\bbE A\x=A\bbE \x$, $\Var(A\x)=A\Sigma A^\TT$, and $\Cov(A\x,B\x)=A\Sigma B^\TT$.
			\item If $\y=A\x+\b$, where $A\in\bbR^{q\times p}$, $\b\in\bbR^{q}$, then $\bmu_\y=A \bmu_\x+\b$ and $\Sigma_\y=A\Sigma_\x A^\TT$.     
		\end{itemize}  
		Let $\Z\in\bbR^{p\times q}$ be a random matrix, $B\in\bbR^{m\times p}$, $C\in\bbR^{q\times n}$, and $D\in\bbR^{m\times n}$ constants, then 
		\begin{itemize}
			\item $\bbE (B\Z C+D) = B \bbE (\Z) C+D$. 
		\end{itemize}   
	\end{proposition}
\end{thmbox}
\begin{itemize}
	\item The $\Sigma\in\bbR^{p\times p}$ is a covariance matrix (i.e., $\Sigma=\Cov(\x)$ for some random vector $\x\in\bbR^p$) iff $\Sigma\succeq \0$.
	\begin{itemize}
		\item \small $(\Leftarrow)$: suppose $\rmr(\Sigma)=r\leq p$, write full rank decomposition $\Sigma=C C^\TT$, $C\in\bbR^{p\times r}$. Let $\y\sim[\0_r,I_r]$, then $\Cov(C\y)=\Sigma$.    
	\end{itemize}
	\item If $\Sigma$ is not PD, then $\exists \a\neq\0_p$ s.t. $\Var(\a^\TT \x)=0$ so w.p.1., $\a^\TT \x=k$, i.e., $\x$ lies in a hyperplane.     
\end{itemize}

\begin{thmbox}
	\begin{theorem}\label{thm:rv_uni_linearFun}
		If $\x\in\bbR^p$ random, then its distribution is uniquely determined by the distributions of $\a^\TT \x$, $\forall \a\in\bbR^p$.  
	\end{theorem}
\end{thmbox}
The proof uses the fact that a distribution in $\bbR^p$ is uniquely determined by its ch.f., see Theorem 1.2.2. \cite{muirhead1982aspects}.



\begin{defbox}
	\begin{definition}\label{def:sample_mean_etc}
		Dataset contains $p$ variables and $n$ observations are represented by 
		$X = (\x_{1},\ldots,\x_n)^\TT$, where the $i$th row $\x_i^\TT=(x_{i1},\ldots,x_{ip})$ is the $i$th observation vector, $i=1,\ldots,n$. 
		\begin{itemize}
			\item (Sample mean vector) $\bar{\x}=n^{-1}\sum_{i=1}^{n}\x_i=(\bar{x}_1,\ldots,\bar{x}_p)^\TT$, where $\bar{x}_j=n^{-1}\sum_{i=1}^{n}x_{ij}$.
			\item (Sum of squares and cross product (SSCP) matrix) $A=\sum_{i=1}^{n}(\x_i-\bar{\x})(\x_i-\bar{\x})^\TT$.
			\item (Sample covariance matrix) $S=(n-1)^{-1}A$.
			\item (Sample correlation matrix) $R=D^{-1/2}S D^{-1/2}$, where $D^{-1/2}=\diag(1/\sqrt{s_{11}},\ldots,1/\sqrt{s_{pp}})$.
		\end{itemize}
	\end{definition}
\end{defbox}
\begin{itemize}
	\item $\bar{\x}=n^{-1}X^\TT\1_n$, $A=(X-\1_n\bar{\x}^\TT)^\TT(X-\1_n\bar{\x}^\TT)\succeq \0$.
	\item $\bbE\bar{\x}=\bmu$, $\Var(\bar{\x})=n^{-1}\Sigma$, $\bbE A=(n-1)\Sigma$, and $\bbE S=\Sigma$.
\end{itemize}

\subsection{Multivariate normal distribution}\label{sec:mult_normal}
\begin{defbox}
	\begin{definition}[Original definition of multivariate normal]\label{def:oridef_multi_normal}
		The random vector $\x\in\bbR^p$ is said to have an $p$-variate normal distribution ($\x\sim\rmN_p$) if $\forall \a\in\bbR^p$, the distribution of $\a^\TT\x$ is univariate normal.  
	\end{definition}
\end{defbox}
\begin{thmbox}
	\begin{theorem}[Fundamental properties]\label{thm:multi_normal}
		Let $\x\sim\rmN_p$, we have
		\begin{enumerate}
			\item Both $\bmu=\bbE\x$ and $\Sigma=\Cov(\x)$ exist and the distribution of $\x$ is determined by $\bmu$ and $\Sigma$. Write $\x\sim\rmN_p(\bmu,\Sigma)$. 
			\item (\red{Representation}) Let $\Sigma\succeq\0_{p\times p}$, $\rmr(\Sigma)=r\leq p$, and $u_{1:r}\simiid\rmN(0,1)$, i.e., $\u\sim\rmN_r(\0_r,I_r)$, then if $C$ is the full rank decomposition of $\Sigma$ and $\bmu\in\bbR^p$, then $\x=C\u+\bmu\sim\rmN_p(\bmu,\Sigma)$.  
			\begin{itemize}
				\item Let $\Sigma=HDH^\TT$ be the spectral
				decomposition, then $\x=HD^{1/2}\z+\bmu$, where $\z\sim\rmN_p(\0_p,I_p)$.
			\end{itemize}
			\item If $\x\sim\rmN_p(\bmu,\Sigma)$, then its \red{ch.f.} $\phi_\x(\t)=\exp(i\bmu^\TT\t-\t^\TT\Sigma\t/2)$.
			\item (\red{Density}) $\x\sim\rmN_p(\bmu,\Sigma)$ with $\Sigma\succ \0$, then $\x$ has pdf
			\begin{sequation}\label{eq:mult_normal_density}
				f(\boldsymbol{x})=\frac{1}{(2\pi)^{p/2}|{\Sigma}|^{1/2}}\exp\left\{-\frac{1}{2}(\boldsymbol{x}-\boldsymbol{\mu})^{T}{\Sigma}^{-1}(\boldsymbol{x}-\boldsymbol{\mu})\right\}.
			\end{sequation}
		\end{enumerate}
	\end{theorem}
\end{thmbox}
Note that we guarantee the existence of $\rmN_p(\bmu,\Sigma)$ by means of the representation in point 2. 

\begin{thmbox}
	\begin{theorem}[Properties of multivariate normal]\label{thm:multi_normal_prop}
		If $\x\sim\rmN_p(\bmu,\Sigma)$, then we have 
		\begin{enumerate}
			\item (\red{Linearity}) Let $B\in\bbR^{q\times p}, \b\in\bbR^{q}$ nonrandom, and $B\Sigma B^\TT\succ \0$, then $B\x+\b \sim \rmN_q (B\bmu+\b,B\Sigma B^\TT)$.  
			\item (\red{Linear combinations}) If $\x_k\sim\rmN_p(\bmu_k,\Sigma_k)\indep$ for $k=1,\ldots,N$, then for any fixed constants $\alpha_1,\ldots,\alpha_N$, $\sum_{k=1}^N\alpha_k\x_k\sim\rmN_p(\sum_{k=1}^{N}\alpha_k\bmu_k,\sum_{k=1}^{N}\alpha_k^2\Sigma_k)$.
			\begin{itemize}
				\item The sample mean $\bar{\x}\sim\rmN_p(\bmu,\Sigma/N)$. 
			\end{itemize}
			\item (\red{Subset}) The marginal distribution of any subset of $k(<p)$ components of $\x$ is $k$-variate normal.   
			\item (\red{Marginal distribution}) Partition
			\begin{sequation*}
				\x=\left[{\begin{array}{c}\x_{1}\\\x_{2}\end{array}}\right],\quad\boldsymbol{\mu}=\left[{\begin{array}{c}\bmu_{1}\\\bmu_{2}\end{array}}\right],\quad\boldsymbol{\Sigma}=\left[{\begin{array}{cc}\Sigma_{11}&\Sigma_{12}\\ {\Sigma}_{21}& {\Sigma}_{22}\end{array}}\right], \quad \x_1\in\bbR^{q}, \x_2\in\bbR^{p-q}, \Sigma_{12}\in\bbR^{q\times(p-q)}.
			\end{sequation*}    
			Then $\x_1\sim\rmN_q(\bmu_1,\Sigma_{11})$, $\x_1\indep\x_2$ iff $\Sigma_{12}=\0$. 
			\item (\red{Conditional distribution}) Let $\Sigma_{22}^{-}$ be a generalized inverse of $\Sigma_{22}$ (i.e., $\Sigma_{22}\Sigma_{22}^-\Sigma_{22}=\Sigma_{22}$), then 
			\begin{itemize}
				\item[(a)] $\x_1-\Sigma_{12}\Sigma_{22}^-\x_2\sim \rmN_q(\bmu_1-\Sigma_{12}\Sigma_{22}^-\bmu_2,\Sigma_{11}-\Sigma_{12}\Sigma_{22}^-\Sigma_{21})$, and $\indep \x_2$.
				\item[(b)] $[\x_1\mid\x_2]\sim\rmN_q(\bmu_1+{\Sigma}_{12}{\Sigma}_{22}^-(\x_2-\bmu_2),\Sigma_{11}-\Sigma_{12}\Sigma_{22}^{-}\Sigma_{21})$. 
			\end{itemize}  
			\item (\red{Cramér}) If $p\times 1$ random vectors $\x\indep \y$ and $\x+\y\sim\rmN_p$, then both $\x,\y\sim\rmN_p$.     
		\end{enumerate} 
	\end{theorem}
\end{thmbox}
For point 3, each component of a random vector is (marginally) normal does not imply that the vector has a multivariate normal distribution. Counterexample: let $U_1,U_2,U_3\simiid\rmN(0,1)$, $Z\indep U_{1:3}$. Define 
\begin{sequation*}
	X_1 = \frac{U_1+ZU_3}{\sqrt{1+Z^2}},\quad X_2 = \frac{U_2+ZU_3}{\sqrt{1+Z^2}}.
\end{sequation*}   
Then $[X_1|Z]\sim\rmN(0,1)$, free of $Z$, so $X_1\sim\rmN(0,1)$, and $X_2\sim\rmN(0,1)$. But $(X_1,X_2)$ not normal.
The converse is true if the components of $\x$ are all independent and normal, or if $\x$ consists of independent subvectors, each of which is normally distributed.

For the proof of point 5, we use the lemma: if $\Sigma\succeq \0$, 
then $\ker(\Sigma_{22})\subset \ker(\Sigma_{12})$, and $\im(\Sigma_{21})\subset\im(\Sigma_{22})$. 
So $\exists B\in\bbR^{q\times(p-q)}$ satisfying $\Sigma_{12}=B\Sigma_{22}$.  

\subsection{Basic multivariate distributions}\label{sec:Basic-mult-dist}

\begin{itemize}
 
	
	\item (\red{Sample mean}) If $\x_{1:n}\simiid \rmN_p(\bmu,\Sigma)$, then $\bar{\x}\sim\rmN_p(\bmu,n^{-1}\Sigma)$, and $n(\bar{\x}-\bmu)^{T}\Sigma^{-1}(\bar{\x}-\bmu)\sim\chi_p^2$. The squared generalized distance (Mahalanobis distance) 
	$d_i^2=(\boldsymbol{x}_i-\bar{\boldsymbol{x}})^{T}{S}^{-1}(\boldsymbol{x}_i-\bar{\boldsymbol{x}}) \dto \chi_p^2$.   
	\item \red{MLE} of $(\mu,\Sigma)$ is $(\bar{\x},A/n)$.   

\end{itemize}

\begin{defbox}
	\begin{definition}[Wishart distribution]\label{def:multi_Wishart} 
	\end{definition}
\end{defbox}

\section{Asymptotic properties}\label{sec:asym_multi}
\subsection{Asymptotic distributions of sample means and covariance matrices}\label{sec:asym_multi_sampleMeanCov}
Refer to section 1.2.2, \cite{muirhead1982aspects}.
\begin{thmbox}
	\begin{theorem}[CLT for sample means]\label{thm:CLT_multi_sampleMean_iid}
		Let $\x_{1:n}\simiid[\bmu,\Sigma]$, then 
		\begin{sequation*}
			\sqrt{n}(\bar{\x}_n-\bmu)=\frac{1}{\sqrt{n}}\sum_{i=1}^{n}(\x_i-\bmu)\dto \rmN_p(\0_p,\Sigma). 
		\end{sequation*} 
	\end{theorem}
\end{thmbox}


\bibliographystyle{abbrv}
\bibliography{mybib}
%%% end of doc
\end{document}