\documentclass[10pt,a4paper]{article}
\usepackage[english]{babel}
\usepackage[utf8x]{inputenc}
\usepackage[T1]{fontenc}
\usepackage{scribe}
\usepackage{listings}
\usepackage{eufrak}

\newenvironment{hints}{\textbf{Hints.}}{}

\Scribe{Zhuohua Shen}
\Lecturer{}
\LectureNumber{}
\LectureDate{Nov 2024}
\LectureTitle{Probability and measure}

\lstset{style=mystyle}

\begin{document}
	\MakeScribeTop
	\tableofcontents

%#############################################################
%#############################################################
%#############################################################
%#############################################################

References: STAT5005 and \textit{Probability: Theory and Examples}, 4th edition, by Richard Durrett, published by Cambridge University Press.

\section{Measure Theory}\label{sec:measure}

\subsection{Expectation}\label{sec:measure-expectation}
\begin{lemma}\label{lamma:measure-expectation-expInt}
	Let $X\geq 0$, $p>0$, we have $\E X^p=\int_{0}^{\infty }px^{p-1}\P(X>x) \rmd x$.
\end{lemma}


\section{Law of Large Numbers}\label{sec:LLN}




\subsection{Almost Surely Convergence}\label{sec:LLN-AS}

This lemma gives an equivalent relation between expectation and sum of tail probability.
\begin{lemma}\label{lemma:LLN-AS-expectation-sumTail}
	Let $X_i$ iid and $\varepsilon>0$, then $\sum_{n=1}^{\infty } \P(\abs{X_n}>n\varepsilon) \leq \varepsilon^{-1}\E\abs{X_i} \leq  \sum_{n=0}^\infty \P(\abs{X_n}>n\varepsilon)$.  
\end{lemma}



\bibliographystyle{abbrv}
\bibliography{mybib}
%%% end of doc
\end{document}