\documentclass[10pt,a4paper]{article}
\usepackage[english]{babel}
\usepackage[utf8x]{inputenc}
\usepackage[T1]{fontenc}
\usepackage{scribe}
\usepackage{listings}
\usepackage{eufrak}

\newenvironment{hints}{\textbf{Hints.}}{}

\Scribe{Zhuohua Shen}
\Lecturer{}
\LectureNumber{}
\LectureDate{Nov 2024}
\LectureTitle{Probability and measure}

\lstset{style=mystyle}

\begin{document}
	\MakeScribeTop
	\tableofcontents

%#############################################################
%#############################################################
%#############################################################
%#############################################################

References: STAT5005 and \textit{Probability: Theory and Examples}, 4th edition, by Richard Durrett, published by Cambridge University Press.

\section{Measure Theory}\label{sec:measure}

\subsection{Expectation}\label{sec:measure-expectation}
\begin{lemma}\label{lamma:measure-expectation-expInt}
	Let $X\geq 0$, $p>0$, we have $\bbE X^p=\int_{0}^{\infty }px^{p-1}\bbP(X>x) \rmd x$.
\end{lemma}


\section{Law of Large Numbers}\label{sec:LLN}




\subsection{Almost Surely Convergence}\label{sec:LLN-AS}

This lemma gives an equivalent relation between expectation and sum of tail probability.
\begin{lemma}\label{lemma:LLN-AS-expectation-sumTail}
	Let $X_i$ iid and $\varepsilon>0$, then $\sum_{n=1}^{\infty } \bbP(\abs{X_n}>n\varepsilon) \leq \varepsilon^{-1}\bbE\abs{X_i} \leq  \sum_{n=0}^\infty \bbP(\abs{X_n}>n\varepsilon)$.  
\end{lemma}


\section{Central Limit Theorem}\label{sec:CLT}

\section{Random Walks}\label{sec:RW}
\red{Random walk (RW)}: Let $\X_i$ be iid rvs in $\bbR^d$. Let $\S_n=\sum_{i=1}^{n}\X_i$. Then $\{\S_n:n\geq 1\}$ is called a RW. Take $\S_0=\0$.  

\noindent \red{Simple random walk (SRW)}: If $\bbP(X_i=1)=\bbP(X_i=-1)=1/2$, then $\{S_n\}$ is called a SRW in $\bbR^1$. If $\bbP(\X_i=(1,1))=\bbP(\X_i=(1,-1))=\bbP(\X_i=(-1,1))=\bbP(\X_i=(-1,-1))=1/4$, then called a SRW in $\bbR^2$.

\noindent \underline{\textbf{Long-term behavior of RW}}

\noindent \red{Permutable (or exchangeable)}: An event that does not change under finite permutation of $\{\X_1,\X_2,\ldots\}$.
\begin{itemize}
	\item All events in the tail $\sigma$-field $\calT$ are permutable.
	\item $\{\omega:\S_{n}(\omega)\in B \ \mathrm{i.o.}\}$ is permutable but not tail event. 
	\item $\{\omega:\lim\sup_{n\to\infty}\S_{n}(\omega)/c_{n}\geq1\}$.
\end{itemize}

\begin{thmbox}
\begin{theorem}[Hewitt-Savage 0-1 law]\label{thm:HS01Law}
	If $\X_i$ iid and event $A$ is permutable, then $\bbP(A)=0$ or $1$.   
\end{theorem}
\end{thmbox}

\begin{thmbox}
	\begin{theorem}[Long-term behavior of RW \ref{proof:longTermRW-1d}]\label{thm:longTermRW-1d}
		For a RW in $\bbR$, one of the following has probability $1$:
		\begin{enumerate}[label=(\roman*)]
			\item $S_n=0$ for all $n$ ;
			\item $S_n\to \infty $ as $n\to \infty $;
			\item $S_n\to - \infty $ as $n\to \infty $;
			\item $- \infty = \liminf_{n} S_n < \limsup_n S_n=\infty $.   
		\end{enumerate}  
	\end{theorem}
\end{thmbox}

\noindent \underline{\textbf{For two levels $a<b$, find the probability that RW reaches $b$ before $a$}}

\noindent \red{Filtration}: Let $X_i$ be a sequence of rvs, $\{\calF_n:=\sigma(X_1,\ldots,X_n)\}_{n=1}^\infty $ as an increasing sequence of $\sigma$-fields, is called a filtration. We usually take $\calF_0=\{\phi,\Omega\}$.  

\noindent \red{Stopping time/optional random variable/optimal time/Markov time}: $\tau\in\bbN^+\cup\{\infty \}$ is a stopping time w.r.t. $\{\calF_n\}$ if $\{\tau=n\}\in\calF_n$, $\forall n\in\bbN^+$. (Equivalent def: $\{\tau\leq n\}\in\calF_n$ or $\{\tau\geq n+1\}\in\calF_n$ for $n\in\bbN^+$)
\begin{itemize}
	\item If $\tau_1,\tau_2$ are stopping time, then $\tau_1\wedge\tau_2$, $\tau_1\vee\tau_2$, $\tau_1+\tau_2$ are stopping times.    
\end{itemize}

\appendix
\section{Proofs}\label{sec:proof}
\subsection{Proofs - section \ref{sec:RW}}\label{sec:proof-RW}
\begin{proof}[Proof of Theorem \ref{thm:longTermRW-1d}]\label{proof:longTermRW-1d}
	By the 0-1 law \ref{thm:HS01Law}, $\{\limsup_n S_n\geq c\}$ has probability 0 or 1, meaning that $\limsup_n S_n=c\in [-\infty ,\infty ]$ w.p.1. Since $S_n \eqd S_{n+1}-X_{1}$, we have $c=c-X_1$.
\begin{enumerate}[label=(\roman*)]
	\item If $c\in\bbR$, then $X_1\equiv 0$ a.s., so $S_n=0$ for all $n$ a.s.
\end{enumerate}
If $X_1\neq 0$ a.s., then $c=-\infty $ or $\infty $,   
\begin{enumerate}
	\item[(ii)] If $c=\infty $, and $\liminf_n S_n=\infty $, then case (ii);
	\item[(iii)] If $c=-\infty $, and $\liminf_n S_n=-\infty $, then case (iii);
	\item[(iv)] If $c=\infty $, and $\liminf_n S_n=-\infty $, then case (iv).       
\end{enumerate}
\end{proof}


\bibliographystyle{abbrv}
\bibliography{mybib}
%%% end of doc
\end{document}