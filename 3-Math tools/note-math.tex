\documentclass[10pt,a4paper]{article}
\usepackage[english]{babel}
\usepackage[utf8x]{inputenc}
\usepackage[T1]{fontenc}
\usepackage{scribe}
\usepackage{listings}
\usepackage{eufrak}

\newenvironment{hints}{\textbf{Hints.}}{}

\Scribe{Zhuohua Shen}
\Lecturer{}
\LectureNumber{}
\LectureDate{Nov 2024}
\LectureTitle{Math Tools}

\lstset{style=mystyle}

\begin{document}
	\MakeScribeTop
	\tableofcontents

%#############################################################
%#############################################################
%#############################################################
%#############################################################

References: 
\begin{itemize}
	\item CUHKSZ: MAT3007 - Optimization I
	\item CUHKSZ: MAT3220 - Optimization II. Textbook: 
	\item \begin{enumerate}
		\item \textit{Introduction to Nonlinear Optimization: Theory, Algorithms, and
		Applications with MATLAB}, Amir Beck.
		\item \textit{Convex Optimization}, S. Boyd and L. Vandenberghe.
		\item \textit{Nonlinear Programming}, D. Bertsekas.
		\item \textit{First-Order Methods in Optimization}, Amir Beck
	\end{enumerate}
	
	
\end{itemize}

\section{Nonlinear Optimization}\label{sec:nonlinearOPT}
\subsection{KKT Conditions}\label{sec:KKT}
\begin{thmbox}
	\begin{theorem}[The Fritz-John necessary conditions]\label{thm:KKT-FJnecessary}
		Let $\x^*$ be a local minimum of the problem
		\begin{align*}
		&\min\quad f(\x)\\
		&s.t.\quad g_i(\x)\leq0,\quad i=1,2,\ldots,m
		\end{align*}
		where $f,g_1,\ldots,g_m\in C^1(\R^n)$. Then $\exists$ multipliers $\lambda_0,\ldots,\lambda_m\geq 0$, which are not all zeros, such that  
		\begin{align*}
			\lambda_0\nabla f(\x^*)+\sum_{i=1}^m\lambda_i\nabla g_i(\x^*)& =\0 \\
			\lambda_ig_i(\x^*)& =0, \quad i=1,2,\ldots,m.
		\end{align*} 
	\end{theorem}	
\end{thmbox}

A major drawback of the Fritz-John conditions is, they allow $\lambda_0=0$. Under an additional \blue{regularity condition}, we can assume $\lambda_0=1$. Let $I(\x^*)$ be the set of active constraints at $\x^*$: 
\begin{equation*}
	I(\x^*)=\{i:g_i(\x^*)=0\}.
\end{equation*}
\begin{thmbox}
	\begin{theorem}[The KKT conditions for inequality constrained problems]\label{thm:KKT-ineq_constrained}
		Let $\x^*$ be a local minimum of 
		\begin{align*}
		&\min\quad f(\x)\\
		&s.t.\quad g_i(\x)\leq0,\quad i=1,2,\ldots,m
		\end{align*}
		where $f,g_1,\ldots,g_m\in C^1(\R^n)$. If \blue{$\{\nabla g_i(\x^*)\}_{i\in I(\x^*)}$ are linearly independent}. Then $\exists \lambda_1,\ldots,\lambda_m\geq 0$ such that  
		\begin{align*}
			\nabla f(\x^*)+\sum_{i=1}^m\lambda_i\nabla g_i(\x^*)& =\0 \\
			\lambda_ig_i(\x^*)& =0, \quad i=1,2,\ldots,m.
		\end{align*} 
	\end{theorem}	
\end{thmbox}

\begin{thmbox}
	\begin{theorem}[The KKT conditions for inequality/equality constrained problems]\label{thm:KKT-ineq_eq_constrained}
		Let $\x^*$ be a local minimum of 
		\begin{equation}\label{eq:KKT-ineq_eq}
			\begin{aligned}
				&\min\quad  f(\x)\\
				&s.t.\quad  g_i(\x)\leq0,\quad i=1,2,\ldots,m\\
				&\quad\quad\ h_j(\x)=0,\quad j=1,\ldots,p
				\end{aligned}	
		\end{equation}
		where $f,g_1,\ldots,g_m,h_1,\ldots,h_p\in C^1(\R^n)$. If \blue{$\{\nabla g_i(\x^*),\nabla h_j(\x^*),i\in I(\x^*),j=1,\ldots,p\}$ are linearly independent}. Then $\exists \lambda_1,\ldots,\lambda_m\geq 0$, $\mu_1,\ldots,\mu_p\in\R$,  such that 
		\begin{equation}\label{eq:KKT-point}
			\begin{aligned}
				\nabla f(\x^*)+\sum_{i=1}^m\lambda_i\nabla g_i(\x^*) + \sum_{j=1}^{p}\mu_j\nabla h_j(\x^*)& =\0, \\
				\lambda_ig_i(\x^*)& =0, \quad i=1,2,\ldots,m.
			\end{aligned} 
		\end{equation} 
	\end{theorem}	
\end{thmbox}
Consider problem (1), a feasible point $\x^*$ is called a \red{KKT point}
if $\exists \lambda_1,\ldots,\lambda_m\geq 0$, $\mu_1,\ldots,\mu_p\in\R$, such that \eqref{eq:KKT-point} holds. $\x^*$ is called \blue{regular} if $\{\nabla g_i(\x^*),\nabla h_j(\x^*),i\in I(\x^*),j=1,\ldots,p\}$ are linearly independent.
\begin{itemize}
	\item The additional requirement of regularity is not required in linearly
	constrained problems in which no such assumption is needed.
\end{itemize}








\bibliographystyle{abbrv}
\bibliography{mybib}
%%% end of doc
\end{document}