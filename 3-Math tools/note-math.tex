\documentclass[10pt,a4paper]{book}
\usepackage[english]{babel}
\usepackage[utf8x]{inputenc}
\usepackage[T1]{fontenc}
\usepackage{scribe}
\usepackage{listings}
\usepackage{eufrak}

\newenvironment{hints}{\textbf{Hints.}}{}

\Scribe{Zhuohua Shen}
\Lecturer{}
\LectureNumber{}
\LectureDate{Nov 2024}
\LectureTitle{Math Tools}

\lstset{style=mystyle}

\begin{document}
	\MakeScribeTop
	\tableofcontents

%#############################################################
%#############################################################
%#############################################################
%#############################################################

\chapter{Complex Analysis}\label{chap:complex}
References: 
\begin{itemize}
	\item CUHKSZ: MAT3253 - Complex Variables notes by Kenneth Shum (2022-2023 Spring)
\end{itemize}

\section{Complex Numbers}\label{sec:complex-num}
\underline{\red{Polar form of complex numbers}} $z=x+iy=r(\cos\theta+i\sin\theta)$ for $r,\theta \geq 0$.
\begin{itemize}
	\item If $z_k=r_k(\cos\theta_k+i\sin\theta_k)$ for $k=1,2$, then $z_1z_2=r_1r_2[\cos(\theta_1+\theta_2)+i\sin(\theta_1+\theta_2)]$.   
	\item If $z_1z_2z_3=0$, then at least one of the three factors is zero. 
	\item If $\Re(z_1),\Re(z_2)>0$, then $\Arg(z_1z_2)=\Arg(z_1)+\Arg(z_2)$, where principal arguments in $(-\pi,\pi]$ are used.    
\end{itemize}   
\underline{\red{Properties of complex numbers}} (i) $(z^*)^*=z$; (ii) $z^*=z$ iff $z\in\bbR$; (iii) $zz^*=\abs{z}^2=x^2+y^2$; (iv) $z_1,z_2\in\bbC$, $(z_1+z_2)^*=z_1^*+z_2^*, (z_1z_2)^*=z_1^*z_2^*$; (v) $\Re(z)=(z+z^*)/2, \Im(z)=(z-z^*)/(2i)$; (vi) $\abs{z_1+z_2}\leq \abs{z_1}+\abs{z_2}$; (vii) $z_1\neq z_2$, then $\abs{z_2-z_1}^2 = r_1^2 + r_2^2 - 2r_1r_2\cos(\theta_1-\theta_2)$; (viii) $\abs{z_1+z_2}^2\leq \abs{z_1}^2 + 2\abs{\Re(z_1z_2^*)} + \abs{z_2}^2$, and $\abs{\Re (z_1z_2^*)}\leq\abs{z_1}\abs{z_2}$.
\begin{itemize}
	\item (\red{DeMoivre formula}) $\forall n\in\bbZ, \theta\in\bbR$, $(\cos\theta+i\sin\theta)^n=\cos(n\theta)+i\sin(n\theta)$ 
	\item (\red{Binomial formula}) $(z_1+z_2)^m=\sum_{k=0}^{m}\binom{m}{k}z_1^kz_2^{m-k}$ for $m\in\bbN^+$, $z_1,z_2\in\bbC$.   
	\item (Geometric series) $\sum_{k=0}^{n}z^k=(1-z^{n+1})/(1-z)$.
\end{itemize}
\underline{\red{$n$-th root of a complex number}} $w$ is the $n$-th root of $z_0$ if $w^n=z_0$.    
\begin{itemize}
	\item ($n$-th root of unity) $\forall n\in\bbN^+$, the solution of $z^n=1$ is $z=\cos(2\pi k/n)+i\sin(2\pi k/n)$, $k=0,\ldots,n-1$.
	If we write $w=\cos(2\pi/n)+i\sin(2\pi/n)$, then the $n$-th root is $w^k$, $k=0,\ldots,n-1$.       
\end{itemize}

\begin{exbox}
\begin{example}\
	\begin{itemize}
		\item (Summation of $\cos k\theta$)
		\begin{sequation*}
			1+\cos\theta+\cos 2\theta+\cdots+\cos n\theta=\frac{1}{2}+\frac{\sin((2n+1)\theta/2)}{2\sin(\theta/2)}, \quad 0<\theta<2\pi.
		\end{sequation*}
		\item (Chebyshev polynomials) Let $m=n/2$ if $n$ is even and $(n-1)/2$ if $n$ is odd, then    
		\begin{sequation*}
			\cos n\theta = \sum_{k=0}^{m} \binom{n}{2k}(-1)^k \cos^{n-2k}(\theta)\sin^{2k}(\theta), \quad n\in\bbN.
		\end{sequation*}
		 Write $x=\cos\theta$, the above becomes a polynomial $T_n(x)$ of degree $n$ in the variable $x$.   
	\end{itemize}
	
\end{example}
\end{exbox}

\subsection{Transformation}\label{sec:complex-trans}
\underline{\red{Linear fractional/Möbius/bilinear transformation}} 
\begin{sequation*}
	f(z)=\frac{az+b}{cz+d}=\frac{a}{c}+\frac{bc-ad}{c}\frac{1}{cz+d}, \quad a,b,c,d\in\bbC,\ ad-bc\neq 0.
\end{sequation*}
\begin{itemize}
	\item $b=0,c=0,d=1$, rotation $f(z)=az=re^{i\theta}z$;
	\item $a=1,c=0,d=1$, translation $f(z)=z+b$;
	\item $a=0,b=1,c=1,d=0$, inversion function $f(z)=1/z$, that maps circles and straight lines to circles and straight lines;
	\item $f(z)=rz$, $0<r\in\bbR$, scaling.  
\end{itemize}  
All four types of transformation maps circle/line to circle/line.
If $ad-bc=0$, then $f(z)$ is a constant.

When $z=-d/c$, $f(z)=\infty $, we extend the domain.
The \red{Riemann sphere} is three-dimensional sphere with the south pole
touching the origin of the complex plane. The \red{stereographic projection} s a function that
maps a complex number $z=x+iy$ in the complex plane to the point $P(x,y)$ on the Riemann
sphere such that $(x,y),P(x,y)$ and the north pole of the sphere are colinear. The north
pole of the sphere does not correspond to any point on the complex plane and is called the \red{point at infinity}, and is denoted by the symbol $\infty $. The Riemann sphere is often called the \textit{one-point compactification} of the complex plane.

\noindent\underline{\red{Extended complex number system/extended complex plane}} $\bar{\bbC}, \hat{\bbC}=\bbC\cup\{\infty \}$.  
\begin{itemize}
	\item Given complex numbers $a,b,c,d\in\bbC$ with $ad-bc\neq 0$, define a linear fractional transformation on $\bar{\bbC}$ by   
	\begin{sequation*}
		f(z)=\begin{cases}
			\frac{az+b}{cz+d}&\mathrm{if }z\neq-d/c,z\neq\infty\\
			\infty&\mathrm{if }z=-d/c\\
			a/c&\mathrm{if }z=\infty
		\end{cases}
	\end{sequation*}
	which is a bijection on the Riemann sphere.
\end{itemize}

\section{Complex functions}\label{sec:complex-fun}
\subsection{Complex sequences and series}\label{sec:complex-seq}
\begin{itemize}
	\item Distance $d(z_1,z_2)=\abs{z_1-z_2}$.
	\item \red{Open disc} of radius $r$ centered at $z_0$: $D(z_0,r)=\{z\in\bbC : \abs{z-z_0}<r\}$. Neighborhood of $\infty $ is $\{z\in\bbC:\abs{z}>R\}$ for some large $R$.   
	\item \red{Convergence} of complex sequence $(z_n)_{n=1}^\infty $: converges to $L\in\bbC$ if $\forall \epsilon>0, \exists N\in\bbN $ s.t. $\abs{z_n-L}<\epsilon$, $\forall n>N$. We write $\lim_{n\to \infty }z_n=L$. $(z_n)_{n=1}^\infty$ converges to $\infty $ iff $1/\abs{z_n}\to 0$ as $n\to \infty $ ($\abs{z_n}\to \infty $). It \red{diverges} if $z_n$ does not converge to any $L\in\bbC$ ($\to \infty $ is also divergent for $\bbC$).
	\item \red{Cauchy sequence} $\forall \epsilon>0,\exists N\in\bbN$ s.t. $\abs{z_m-z_n}\leq \epsilon$, $\forall m,n\geq N$.
	\begin{itemize}
		\item If $z_n=x_n+i y_n$, then $z_n$ is Cauchy iff $x_n,y_n$ Cauchy.
		\item $z_n$ converges iff $z_n$ is Cauchy.     
	\end{itemize}                
	\item \red{Complex series} $\sum_{k=1}^{\infty }z_k :=\lim_{n} \sum_{k=1}^{n}z_k$ if the limit exists. We call it \red{converges absolutely} if $\sum_{k=1}^{\infty }\abs{z_k}$ converges. We call it \red{converges conditionally} if $\sum_{k=1}^{\infty }z_k$ converges but $\sum_{k=1}^{\infty }\abs{z_k}$ diverges.  
	\begin{itemize}
		\item \red{($n$-th term test)} If $\sum_{k=1}^{\infty }z_k$ converges, then $\lim_{n}\abs{z_n}=0$. If $\abs{z_n}\nrightarrow 0$, then $\sum_{k}z_k$ diverges.
		\item \red{(Absolute convergence test)} $\sum_{k=1}^{\infty }\abs{z_k}$ converges, then $\sum_{k=1}^{\infty }z_k$ converges.     
		\item \red{(Limit ratio test)} Assume $\lim_{n}\abs{a_{n+1}/a_n}$ exists and is equal to $L$. (a) $L>1\Rightarrow\sum_{k=1}^{\infty }a_n $ diverges, (b)  $L<1\Rightarrow\sum_{k=1}^{\infty }a_n $ converges absolutely, (c) $L=1$, no conclusion.
		\item If $\sum_{k=0}^{\infty }a_k$ and $\sum_{k=0}^{\infty }b_k$ converges absolutely, then $(\sum_{k=0}^{\infty }a_k)(\sum_{k=0}^{\infty }b_k)=(\sum_{k=0}^{\infty }c_k)$, where $c_k=\sum_{j=0}^{k}a_j b_{k-j}$.
		\item If a series converges absolutely, then a series obtained by rearranging the terms converges to the same limit.    
	\end{itemize} 
\end{itemize}

\subsection{Basic complex functions}
\noindent\underline{\red{Power series}} A complex power series centered at the origin is a series in the form $\sum_{k=0}^{\infty }a_kz_k^k$, $\a_k\in\bbC$.
\begin{defbox}
\begin{definition}\label{def:complex-fun-power}
	For $\z\in\bbC$, define 
	\begin{itemize}
		\item (\red{complex exponential function})
		\begin{sequation*}
			e^z :=\exp(z):=\sum_{n=0}^{\infty }\frac{z^n}{n!}, 
		\end{sequation*}
		we have $e^{z_1+z_2}=e^{z_1}e^{z_2}$, $e^{-z}=(e^{z})^{-1}$, $e^z\neq 0$, $\forall z\in\bbC$, and $e^{a+ib}=e^a e^{ib}$, $a,b\in\bbR$.      
		\item (\red{complex trigonometric, hyperholic trigonometric})
		\begin{sequation*}
			\begin{aligned}
				\sin(z)&:=\sum_{n=0}^\infty(-1)^n\frac{z^{2n+1}}{(2n+1)!}, \quad
				\cos(z):=\sum_{n=0}^\infty(-1)^n\frac{z^{2n}}{(2n)!}, \quad \tan(z):=\frac{\sin(z)}{\cos(z)}, \\
				\sinh(z)&:=\sum_{n=0}^{\infty}\frac{z^{2n+1}}{(2n+1)!}, \qquad\quad
				\cosh(z):=\sum_{n=0}^{\infty}\frac{z^{2n}}{(2n)!}, \qquad\ \  \tanh(z):=\frac{\sinh (z)}{\cosh (z)}.
			\end{aligned}
		\end{sequation*}
	\end{itemize}
	They are all converges absolutely. 
\end{definition}
\end{defbox} 

\begin{thmbox}
	\begin{theorem}\label{thm:Euler_formula} $\forall z\in\bbC$, 
		\begin{itemize}
			\item (\red{Euler's formula}) $e^{iz} = \cos z + i\sin z$
			\item \begin{sequation*}
			\begin{aligned}
				\cos z &= \frac{e^{iz}+e^{-iz}}{2}, \quad \sin z = \frac{e^{iz}-e^{-iz}}{2i}, \\
				\cosh (z) &= \frac{e^z+e^{-z}}{2}, \quad \sinh (z) = \frac{e^{z}-e^{-z}}{2}, 
			\end{aligned}  
			\end{sequation*}
			thus we have $\cosh(iz)=\cos z$, $\sinh(iz)=i \sin z$. 
		\end{itemize}
	\end{theorem}
\end{thmbox}

\subsubsection{Complex logarithm and complex power}
Since $e^{z}=e^{z+2\pi ki}, k\in\bbZ$, the inverse function of $e^z$ is multi-valued. For $0\neq w\in\bbC$ , define the \red{complex log function} as 
\begin{sequation*}
	\log (w) := \log\abs{w} + i(\arg(w)+2\pi k), \ k\in\bbZ.
\end{sequation*} 
Define the \red{principal complex log function} as 
\begin{sequation*}
	\Log (w) := \log \abs{w} + i \arg(w), \ \arg(w)\in (-\pi,\pi] \text{ or } [0,2\pi).
\end{sequation*}
Given $0\neq z\in\bbC$, define the \red{complex power} by 
\begin{sequation*}
	z^w := \exp(w \log(z)).
\end{sequation*} 

\subsubsection{The angle function, parametric curve and winding number}
Suppose $\theta(z)$ is continuous, $\theta(z_0)=0$, then as $z\to z_0$ from the right, $\theta(z)\to 2\pi\neq 0$. 
To prevent closed cycle around the origin, let the domain of \red{angle function} be the half plane. 
\begin{itemize}
	\item If $H=\{x+iy:y>0\}$, the range is $(0,\pi)$, the for $z\in H$, define $F(x,y) := \cos^{-1}({x}/{\sqrt{x^2+y^2}})$.
	\item If $H_\alpha=\{x+iy: y>\tan(\alpha) x\}$ for $\alpha>0$, define $F_\alpha := F(e^{-i\alpha}z)+\alpha$.     
\end{itemize}    

The \red{parametric curve} is a function $\gamma:[a,b]\to\bbC$ continuous, $\gamma(t)$ is the location at time $t$.
If $\gamma(a)=\gamma(b)$, then we call it \red{closed curve}.    

Given $\gamma:[a,b]\to \bbC\setminus\{0\}$, divide $[a,b]$ into $a=t_0<t_1<\cdots<t_{n-1}<t_n =b$ s.t. $\gamma(t)$, $t_k\leq t\leq t_{k+1}$, is inside $H_{\alpha(k)}$ for $k=0,\ldots,n-1$.
Define 
\begin{itemize}
	\item \red{change in angle} in the $k$th part: $F_{\alpha(k)}(\gamma(t_{k+1}))-F_{\alpha(k)}(\gamma(t_k))$, 
	\item \red{overall change of angle}: $\sum_{k=0}^{n-1}[F_{\alpha(k)}(\gamma(t_{k+1}))-F_{\alpha(k)}(\gamma(t_k))]$,
	\item \red{branch}: A continuous angle function as a function of $t$. 
\end{itemize}
Note that it doesn't depend on the sub-division of the curve and how we parameterize the curve.
The \red{winding number/index} of a closed parametric curve not passing through the origin is $(2\pi)^{-1}\dot(\text{change in angle})$. 

\subsection{Complex differentiability}\label{sec:complex-diff}
\subsubsection{Limit and continuity}

\chapter{Linear Algebra}\label{chap:linear_algebra}
References: 
\begin{itemize}
	\item CUHKSZ: MAT2040 - Linear Algebra, by Dr. Dongxu Ji (2020-2021 Summer).
	\item CUHK: STAT5030 - Linear Models, by Prof. Yuanyuan Lin (2024-2025 Spring)
	\item Peng Ding - \textit{Linear Model and Extensions}: appendix A.
	\item Robb J. Muirhead - \textit{Aspects of multivariate statistical theory} \cite{muirhead1982aspects}: appendix A.
\end{itemize}

\noindent Notations:  
\begin{itemize}
	\item \red{Column space/range/image} of $n\times m$ matrix $A=(\a_1,\ldots,\a_m)$ is $\Col(A)=\{\alpha_{1}\a_{1}+\cdots+\alpha_{m}\a_{m}:\alpha_{1},\ldots,\alpha_{m}\in\bbR\}$.
	\item \red{Row space} of $A$ is $\Col(A^\TT)$.
	\item \red{Kernel/null space/nullspace} of $A$ is $\ker(A)=\{\v\in\bbR^{m}:A\v=\0_n\}$.  
	\item \red{Rank} of $A$ is $\rmr(A)$.  
	\item \red{Trace} of $A$ is $\tr(A)$.  
	\item \red{Eigenvalues} of $A$ are $\lambda(A)$.  
\end{itemize}

\section{Spaces}\label{sec:mat_space}
\subsection{Spaces, rank, and related factorization}\label{sec:mat_space_rank_fact}
\begin{thmbox}
	\begin{theorem}[Properties of rank]\label{thm:mat_rank}
		Let $A\in\bbR^{n\times m}$. 
		\begin{enumerate}
			\item $\rmr(A)=\rmr(A^\TT)$.
			\item $\rmr(A)\leq\min(n,m)$.   
			\item $\rmr(A)=\rmr(A^\TT A)$. 
			\item $\rmr(AB)\leq \min\{\rmr(A),\rmr(B)\}$.
			\item If $A,B$ have same size, then $\rmr(A+B)\leq\rmr(A)+\rmr(B)$.
			\item If $A\in\bbR^{n\times n},B\in\bbR^{n\times m},C\in\bbR^{m\times m}$, and $A$ and $C$ are nonsingular, then $\rmr(ABC)=\rmr(B)$. 
			\item If $B\in\bbR^{m\times p}$ such that $AB=\0$, then $\rmr(B)\leq m-\rmr(A)$.    
			\item $A$ is full column/row rank iff $A^\TT A$/$AA^\TT$ is nonsingular.  
			\item The system of equations $A\x=\c$ is consistent iff $\rmr(A)=\rmr([A,\c])$.  
		\end{enumerate}
	\end{theorem}
\end{thmbox}

\begin{thmbox}
	\begin{theorem}[Rank-related factorization]\label{thm:mat_rank_fact}
		\
		\begin{enumerate}
			\item (\red{Full-rank factorization}) If $A\in\bbR^{n\times m}$ with $\rmr(A)=k$, then $A=BC$ for some full column rank $B\in\bbR^{n\times k}$ and full row rank $ C\in\bbR^{k\times m}$.    
			\item (\red{Non-negative definite}) If $n\times n$ $A\succeq \0$, $\rmr(A)=r$, then 
			\begin{itemize}
				\item $\exists B\in\bbR^{n\times r}$ of rank $r$ such that $A=BB^\TT$;
				\item $\exists C\in\bbR^{n\times n}$ nonsingular such that $A=C\begin{bmatrix}
					I_r & \0 \\ \0 & \0
				\end{bmatrix} C^\TT$.     
			\end{itemize}
		\end{enumerate}
	\end{theorem}
\end{thmbox}

\section{Inversion}\label{sec:mat_inv}

\subsection{Types of inversion}\label{sec:mat_inv_types}
\subsubsection{Left and right inverse}\label{sec:LR_inv}
\begin{defbox}
	\begin{definition}[Left/right inverse]\label{def:LR_inv}
	Let $A\in\bbR^{n\times m}$. The \red{left inverse} $A_{left}^{-1}\in\bbR^{m\times n}$ satisfies $A_{left}^{-1} A = I_m$, and the \red{right inverse} $A_{right}^{-1}\in\bbR^{m\times n}$ satisfies $A A_{right}^{-1}=I_n$. 
	\end{definition}
	  
\end{defbox}
We can find these inverses by Theorem \ref{thm:mat_rank}: if 
\begin{itemize}
	\item $A$ full column rank $\rmr(A)=m$, then $A^\TT A$ nonsingular, and $A_{left}^{-1}=(A^\top A)^{-1}A^\top$; $A^-$ is also a left inverse. 
	\item $A$ full row rank $\rmr(A)=n$, then $AA^T$ nonsingular, and $A_{right}^{-1}=A^{\top}(AA^{\top})^{-1}$; $A^-$ is also a right inverse.
\end{itemize}
   

\subsubsection{Moore--Penrose inverse}\label{sec:MP_inv}
It is used when left/right inverse cannot be obtained.
\begin{defbox}
	\begin{definition}[\href{https://en.wikipedia.org/wiki/Moore–Penrose_inverse}{Moore--Penrose inverse/pseudoinverse}]\label{def:MP_inv}
		Let $A\in\bbR^{n\times m}$. $A^+\in\bbR^{m\times n}$ is defined as a \red{Moore-Penrose (M-P) inverse} of $A$ if
		\begin{enumerate}
			\item $A A^+$ and $A^+ A$ are symmetric;
			\item $A A^+ A=A$;
			\item $A^+ A A^+=A^+$.    
		\end{enumerate}   
	\end{definition}
\end{defbox}

\begin{exbox}
	\begin{example}[M-P inverse of a diagonal matrix]\label{ex:MP_diag}
		Let $D\in\bbR^{n\times m}$, WLOG, assume $n\geq m$. $D_{ii}=d_i$ for $i=1,\ldots,m$, while others are zero. Then $D^+\in\bbR^{m\times n}$ satisfies $D_{ii}^+=1/d_i$ if $d_i\neq 0$ and zero otherwise, $i=1,\ldots,m$.        
	\end{example}
\end{exbox}

\begin{thmbox}
	\begin{theorem}\label{thm:exist_MP_inv}
		Each matrix $A$ has an $A^+$.  
	\end{theorem}
\end{thmbox}
If $A=\0$, then $A^+=\0$. If $A\neq\0$, two ways to construct $A^+$:
\begin{enumerate}
	\item If $A$ is symmetric with eigendecomposition $A=P^\TT D P$, then $A^+=P^\TT D^+ P$, where $D^+$ is given in Ex. \ref{ex:MP_diag}.  
	\item Full-rank factorization $A=B_{n\times r}C_{r\times m}$ of rank $r$. We have 
	\begin{sequation*}
		A^+=C^\top(CC^\top)^{-1}(B^\top B)^{-1}B^\top.
	\end{sequation*} 
	\item SVD $A=UDV^\TT$, then $A^+=VD^+U^\TT$. 
\end{enumerate}  
 

\begin{thmbox}
	\begin{theorem}\label{thm:prop_MP_inv}
		Let $A\in\bbR^{n\times m}$. 
		\begin{enumerate}
			\item The M-P inverse is unique.
			\item $(A^{\top})^{+}=(A^{+})^{\top}$.
			\item $\rmr(A^{+})=\rmr(A)$.
			\item If $A$ is symmetric, then $A^{+}=(A^{+})^{\top}$.
			\item If $A$ is nonsingular, then $A^{-1}=A^{+}$.
			\item If $A$ is symmetric idempotent, then $A^{+}=A$.
			\item If $r(A)=m$, then $A^{+}=A_{lrft}^{-1}=(A^{\top}A)^{-1}A^{\top}$.
			\item If $r(A)=n$, then $A^{+}=A_{right}^{-1}=A^{\top}(AA^{\top})^{-1}$.
			\item The matrices $AA^{+},A^{+}A,I_n-AA^{+}$, and $I_m-A^{+}A$ are all symmetric idempotent.
			\item $AA^+$ ($A^+ A$) is a p.p.m. onto $\Col(A)$ ($\Col(A^\TT)$).  
		\end{enumerate}
	\end{theorem}
\end{thmbox}

\subsubsection{Generalized Inverse}\label{sec:gen_inv}
\begin{defbox}
	\begin{definition}[\href{https://en.wikipedia.org/wiki/Generalized_inverse}{Generalized inverse}]\label{def:gen_inv}
		Let $A\in\bbR^{n\times m}$. The \red{generalized inverse (G-inverse)} $A^-\in\bbR^{m\times n}$ satisfies 
		\begin{sequation*}
			A A^- A = A.
		\end{sequation*}
	\end{definition}
\end{defbox}
It is obvious that the M-P inverse is also a G-inverse. 
$\forall A$, G-inverse exists.
\begin{enumerate}
	\item If $A$ nonsingular, then only $A^-=A^{-1}$.
	\item If $A=\0_{n\times m}$, then $A^{-}=\0$.
	\item If $D$ is diagonal like Example \ref{ex:MP_diag}, then let $D^{-}=D^{+}$.
	\item If $A$ is symmetric with eigendecomposition $A=P^\TT D P$, then let $A^-=A^+$, symmetric.    
	\item If $\rmr(A)=r$, then by SVD, $A=U_{n\times n}\begin{bmatrix}
		\Sigma_{r\times r} & \0 \\ \0 & \0
	\end{bmatrix}V_{m\times m}^\TT$. Let $A^{-}=V\begin{bmatrix}
		\Sigma^{-1} & B \\ C & D
	\end{bmatrix}U^\TT$, $\forall B,C,D$. 
	\item If $G_1,G_2$ are G-inverse of $A$, then so is $G_1AG_2$.   
\end{enumerate}  
Note that 
\begin{itemize}
	\item G-inverse may not be unique by the above construction. 
	\item $A$ is symmetric, $A^-$ may not be symmetric.
\end{itemize}

\begin{thmbox}
	\begin{theorem}[Properties of G-inverse]\label{thm:gen_inv}
		Let $A\in\bbR^{n\times m}$ with rank $k>0$.
			\begin{enumerate}
				\item $\rmr( A^- ) \geq k.$
				\item $A^-A$ and $AA^-$ are idempotent.
				\item $\ker(A^- A)=\ker(A)$ and $\Col(AA^-)=\Col(A)$. 
				\item $\rmr( A^- A) = \rmr( AA^- ) = k.$
				\item $A^- A= I_m$ (i.e., $A^-$ is a left inverse of $A$) iff $\rmr(A)=m$.  
				\item $AA^-=I_n$ (i.e., $A^-$ is a right inverse of $A$) iff $\rmr(A)=n$.
				\item $\tr(A^-A)=\tr(AA^-)=k=\rmr(A)$.
				\item If $A^-$ is any G-inverse of $A$, then $(A^-)^\top$ is a G-inverse of $A^\top$.
				\item The system of equations $A\x=\c$ is consistent iff $\forall A^-$ of $A$, $AA^-\c=\c$ (i.e. $A^-\c$ is a solution).    
				\item If $G,H$ are G-inverses of $(A^\TT A)$, then 
				\begin{enumerate}
					\item $AGA^\TT A=AHA^\TT A=A$, i.e., $A(A^\TT A)^{-}A^\TT A=A$ for any G-inverse $(A^\TT A)^{-}$.
					\begin{itemize}
						\item $(A^\top A)^-A^\top$ is a G-inverse of $A$ for any G-inverse of $A^\top A.$
						\item $A(A^\top A)^-$ is a G-inverse of $A^\top$ for any G-inverse of $A^\top A.$
					\end{itemize}
					\item $AGA^\TT=AHA^\TT$, i.e., $A(A^\TT A)^{-}A^\TT$ the same for any G-inverse $(A^\TT A)^{-}$.
					\item Since $A^\TT A$ is symmetric, $\exists (A^\TT A)^{-}$ symmetric such that $A(A^\TT A)^{-}A^\TT$ symmetric. So by ii, $A(A^\TT A)^{-}A^\TT$ symmetric for all $(A^\TT A)^{-}$.      
					\item $A(A^\TT A)^{-}A^\TT$ is the p.p.m. onto $\Col(A)$, see Example \ref{ex:ppm_reg}. 
				\end{enumerate}  
			\end{enumerate}
	\end{theorem}
\end{thmbox}

\section{Special matrices}\label{sec:spec_mat}
\subsection{Idempotent matrices}\label{sec:idempotent}
\begin{defbox}
	\begin{definition}[\href{https://en.wikipedia.org/wiki/Idempotent_matrix}{Idempotent matrices}]\label{def:idempotent}
		$A\in\bbR^{n\times n}$ if $A^2=A$.  
	\end{definition}
\end{defbox}

\begin{thmbox}
	\begin{theorem}\label{thm:idempotent}
		$A$ is idempotent.  
		\begin{enumerate}
			\item All idempotent matrices(except $I$) are singular.
			\item $\rmr(A)=\tr(A)$. 
			\item $\lambda(A)$ is either 0 or 1. 
			\item If $A$ symmetric, all $\lambda(A)$'s are 0 or 1, then $A$ is idempotent. 
		\end{enumerate}
	\end{theorem}
\end{thmbox}

\subsection{Perpendicular projection matrices}\label{sec:ppm}
\begin{defbox}
	\begin{definition}[Perpendicular projection matrices]\label{def:ppm}
		$M\in\bbR^{n\times n}$ is a \red{perpendicular projection matrix (p.p.m.)} onto $\Col(X)$ ($X\in\bbR^{n\times m}$) iff 
		\begin{enumerate}
			\item $\forall \v\in\Col(X)$, $M\v=\v$. (Projection - Idempotent)
			\item $\forall \w\in\Col(X)^{\perp}$, $M\w=\0$. (Perpendicularity - Symmetric)  
		\end{enumerate}  
	\end{definition}
\end{defbox}

\begin{thmbox}
	\begin{theorem}[Properties of p.p.m.]\label{thm:prop_ppm}
		Let $M$ be a p.p.m. onto $\Col(X)$ ($X\in\bbR^{n\times m}$). 
		\begin{enumerate}
			\item p.p.m.'s are unique.
			\item $M$ is idempotent and symmetric.
			\item $MX=X$. 
			\item $M$ is a p.p.m. onto $\Col(M)$ iff $M$ is idempotent and symmetric.
			\item Let $\o_1,\ldots,\o_r\in\bbR^{n}$ be orthonormal basis of $\Col(X)$. Let $O=\begin{bmatrix}
				\o_1 & \cdots & \o_r
			\end{bmatrix}$. Then $O O^\TT=\sum_{i=1}^{r}\o_i\o_i^\TT$ is the p.p.m. onto $\Col(X)$.        
		\end{enumerate}
	\end{theorem}
\end{thmbox}

\begin{exbox}
	\begin{example}[Regression]\label{ex:ppm_reg}
		Let $X\in\bbR^{n\times m}$. Then by the property of $(A^\TT A)^{-}$ in Theorem \ref{thm:gen_inv}, $K=X(X^\TT X)^{-}X^\TT\in\bbR^{n\times n}$ is the p.p.m. (also unique) onto $\Col(X)$. 
		We have
		\begin{enumerate}
			\item $K$ is idempotent and symmetric.
			\item $\rmr(K)=\rmr(X)$.
			\item $KX=X$ and $X^\TT K=X^\TT$.  
			\item $K= XX^{+}.$ 
		\end{enumerate}
	\end{example}
\end{exbox}


\chapter{Optimization}\label{chap:opt}

References: 
\begin{itemize}
	\item CUHKSZ: MAT3007 - Optimization I, by Prof. Junfeng Wu (2021-2022 Fall)
	\item CUHKSZ: MAT3220 - Optimization II, by Prof. Shi Pu (2023-2024 Fall). Textbook: 
	\begin{enumerate}
		\item \textit{Introduction to Nonlinear Optimization: Theory, Algorithms, and
		Applications with MATLAB}, Amir Beck.
		\item \textit{Convex Optimization}, S. Boyd and L. Vandenberghe.
		\item \textit{Nonlinear Programming}, D. Bertsekas.
		\item \textit{First-Order Methods in Optimization}, Amir Beck
	\end{enumerate}
\end{itemize}

\section{Nonlinear Optimization}\label{sec:nonlinearOPT}
\subsection{KKT Conditions}\label{sec:KKT}
\begin{thmbox}
	\begin{theorem}[The Fritz-John necessary conditions]\label{thm:KKT-FJnecessary}
		Let $\x^*$ be a local minimum of the problem
		\begin{align*}
		&\min\quad f(\x)\\
		&s.t.\quad g_i(\x)\leq0,\quad i=1,2,\ldots,m
		\end{align*}
		where $f,g_1,\ldots,g_m\in C^1(\bbR^n)$. Then $\exists$ multipliers $\lambda_0,\ldots,\lambda_m\geq 0$, which are not all zeros, such that  
		\begin{align*}
			\lambda_0\nabla f(\x^*)+\sum_{i=1}^m\lambda_i\nabla g_i(\x^*)& =\0 \\
			\lambda_ig_i(\x^*)& =0, \quad i=1,2,\ldots,m.
		\end{align*} 
	\end{theorem}	
\end{thmbox}

A major drawback of the Fritz-John conditions is, they allow $\lambda_0=0$. Under an additional \blue{regularity condition}, we can assume $\lambda_0=1$. Let $I(\x^*)$ be the set of active constraints at $\x^*$: 
\begin{equation*}
	I(\x^*)=\{i:g_i(\x^*)=0\}.
\end{equation*}
\begin{thmbox}
	\begin{theorem}[The KKT conditions for inequality constrained problems]\label{thm:KKT-ineq_constrained}
		Let $\x^*$ be a local minimum of 
		\begin{align*}
		&\min\quad f(\x)\\
		&s.t.\quad g_i(\x)\leq0,\quad i=1,2,\ldots,m
		\end{align*}
		where $f,g_1,\ldots,g_m\in C^1(\bbR^n)$. If \blue{$\{\nabla g_i(\x^*)\}_{i\in I(\x^*)}$ are linearly independent}. Then $\exists \lambda_1,\ldots,\lambda_m\geq 0$ such that  
		\begin{align*}
			\nabla f(\x^*)+\sum_{i=1}^m\lambda_i\nabla g_i(\x^*)& =\0 \\
			\lambda_ig_i(\x^*)& =0, \quad i=1,2,\ldots,m.
		\end{align*} 
	\end{theorem}	
\end{thmbox}

\begin{thmbox}
	\begin{theorem}[The KKT conditions for inequality/equality constrained problems]\label{thm:KKT-ineq_eq_constrained}
		Let $\x^*$ be a local minimum of 
		\begin{equation}\label{eq:KKT-ineq_eq}
			\begin{aligned}
				&\min\quad  f(\x)\\
				&s.t.\quad  g_i(\x)\leq0,\quad i=1,2,\ldots,m\\
				&\quad\quad\ h_j(\x)=0,\quad j=1,\ldots,p
				\end{aligned}	
		\end{equation}
		where $f,g_1,\ldots,g_m,h_1,\ldots,h_p\in C^1(\bbR^n)$. If \blue{$\{\nabla g_i(\x^*),\nabla h_j(\x^*),i\in I(\x^*),j=1,\ldots,p\}$ are linearly independent}. Then $\exists \lambda_1,\ldots,\lambda_m\geq 0$, $\mu_1,\ldots,\mu_p\in\bbR$,  such that 
		\begin{equation}\label{eq:KKT-point}
			\begin{aligned}
				\nabla f(\x^*)+\sum_{i=1}^m\lambda_i\nabla g_i(\x^*) + \sum_{j=1}^{p}\mu_j\nabla h_j(\x^*)& =\0, \\
				\lambda_ig_i(\x^*)& =0, \quad i=1,2,\ldots,m.
			\end{aligned} 
		\end{equation} 
	\end{theorem}	
\end{thmbox}
Consider problem (1), a feasible point $\x^*$ is called a \red{KKT point}
if $\exists \lambda_1,\ldots,\lambda_m\geq 0$, $\mu_1,\ldots,\mu_p\in\bbR$, such that \eqref{eq:KKT-point} holds. $\x^*$ is called \blue{regular} if $\{\nabla g_i(\x^*),\nabla h_j(\x^*),i\in I(\x^*),j=1,\ldots,p\}$ are linearly independent.
\begin{itemize}
	\item The additional requirement of regularity is not required in linearly
	constrained problems in which no such assumption is needed.
\end{itemize}








\bibliographystyle{abbrv}
\bibliography{mybib}
%%% end of doc
\end{document}